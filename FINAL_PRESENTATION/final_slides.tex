\documentclass[9pt,xcolor=table,hyperref=unicode]{beamer}
\setbeamertemplate{background canvas}[vertical shading][bottom=red!10,top=blue!10]
\usetheme{Berkeley}
\usepackage[utf8]{vietnam}
\usepackage{tikz}
\usepackage{hyperref}
\usepackage{booktabs, multicol, multirow}
\usepackage{adjustbox}
\usepackage{array}
\newcolumntype{x}[1]{>{\centering\arraybackslash\hspace{0pt}}p{#1}}
\graphicspath{ {images/} }
\usepackage{xcolor}

\setbeamerfont{page number in head/foot}{size=\tiny}
\setbeamertemplate{footline}[frame number]

\newcommand{\inlineitem}{%
\leavevmode\usebeamertemplate{itemize item}
}
\newcounter{newenumi}
\setcounter{newenumi}{1}

\newcommand{\inlineenum}{%
 {%
 \setcounter{enumi}{\thenewenumi}%
 \leavevmode\usebeamertemplate{enumerate  item}
 \stepcounter{newenumi}
 \setcounter{enumi}{0}
 }
}

\newcommand{\resetinlineenum}{
 \setcounter{newenumi}{1}
}

\begin{document}
	\setbeamertemplate{sidebar left}[sidebar theme]
	
	\title{Luận văn tốt nghiệp}
	\subtitle{Phân giải đồng tham chiếu cho các đối tượng và thuộc tính trong khai khoáng ý kiến}
	\author[]{
		\begin{tabular}{ll}
			Nguyễn Trọng Nghĩa & 51202370 \\
			Nguyễn Đăng Trang & 51203957 \\
			 & 
		\end{tabular}
		\break
		\begin{tabular}{ll}
			GVHD & GS.TS Phan Thị Tươi \\
			GVPB & GS.TS Cao Hoàng Trụ
		\end{tabular}
	}
	\institute{Đại học Bách Khoa TP. Hồ Chí Minh}
	\date{\today}
	
	\begin{frame}
		\Large
		\maketitle
	\end{frame}

	\begin{frame}{Nội dung trình bày}
		\LARGE
		\begin{itemize}			
			\item{Tổng quan đề tài}
			\item{Các công trình liên quan}
			\item{Phương pháp đề xuất}
			\item{Thực nghiệm và đánh giá}
			\item{Tổng kết}
		\end{itemize}
	\end{frame}


	\section{Tổng quan đề tài}
	\begin{frame}
		\frametitle{Tổng quan đề tài}
		\begin{block}{Giới thiệu đề tài}
			\begin{itemize}
				\item{\textit{Phân giải đồng tham chiếu} hướng đến việc tìm kiếm những từ, cụm từ hoặc ngữ cùng chỉ đến một khái niệm, thực thể trong thế giới thực.}
				\item{Nội dung đề tài: \textit{Phân giải đồng tham chiếu} cho \textit{đối tượng} và \textit{thuộc tính} trong các \textit{văn bản chứa ý kiến}.}				
			\end{itemize}
		\end{block}		
		\begin{block}{Ví dụ}			
			\textit{Beckham} will visit Vietnam tomorrow. \textit{He} will attend a football event in Saigon.
		\end{block}			
	\end{frame}
	
	\begin{frame}
		\frametitle{Tổng quan đề tài (tt)}
		\begin{block}{Động cơ thực hiện đề tài}
			\begin{itemize}	
				\item{Thương mại điện tử đang phát triển mạnh mẽ và người dùng ngày càng có nhu cầu thể hiện ý kiến lên các sản phẩm trên mạng.}
				\item{Nếu không có phân giải đồng tham chiếu cho các đối tượng và thuộc tính, ý kiến của người viết rất có thể sẽ được gán không đúng cho các thực thể.}				
				\item{Cải thiện 10\% kết quả Khai khoáng ý kiến (theo Nicolas Nicolov).
				\footnotetext[1]{N. Nicolov, F. Salvetti and S. Ivanova. 2008.
				\textit{Sentiment analysis: Does coreference matter?}.
				In AISB 2008 Convention Communication, Interaction and Social Intelligence.}}
			\end{itemize}		
		\end{block}		
		\begin{block}{Ví dụ}			
			\textit{This Sanyo 8100} is an excellent phone.  \underline{It} has \textit{a very good color screen}, \underline{it} is well designed, \textit{the camera} is pretty good for a phone, and \underline{it} has numerous features.
		\end{block}
	\end{frame}

	\begin{frame}
		\frametitle{Tổng quan đề tài (tt)}
		\begin{block}{Mục tiêu đề tài}
			Tìm ra các từ/cụm từ trong văn bản chứa ý kiến cùng chỉ về một đối tượng hoặc thuộc tính nào đó, tức là tìm các chuỗi đồng tham chiếu của đối tượng và thuộc tính.
		\end{block}		
		\begin{block}{Phạm vi đề tài}
			\begin{itemize}
				\item{Giả định rằng các đối tượng và thuộc tính đã được tìm ra \footnotemark \textsuperscript{,} \footnotemark}
			\end{itemize}
			\footnotetext[1]{Minqing Hu and Bing Liu. 2004.
				\textit{Mining and Summarizing Customer Reviews}.
				In Proceedings of the ACM SIGKDD International Conference on Knowledge Discovery and Data Mining (KDD-2004), Aug 22-25, 2004, Seattle, Washington, USA.
			}
			\footnotetext[2]{A-M. Popescu and O. Etzioni. 2005. 
				\textit{Extracting product features and opinions from reviews}.
				EMNLP’05.
			}
		\end{block}
	\end{frame}


	\section{Các công trình liên quan}
	\begin{frame}
		\frametitle{Các công trình liên quan}
		\begin{block}{Đồng tham chiếu}
			Kể từ những năm 1960 đến nay đã có nhiều công trình nghiên cứu, với nhiều mô hình, nhiều hướng tiếp cận giải quyết khác nhau.
			\begin{itemize}
				\item{Các mô hình: Mô hình cặp, mô hình hướng thực thể, mô hình xếp hạng}
				\item{Các hướng tiếp cận: Học có giám sát, học không giám sát, hệ thống luật}
			\end{itemize}
		\end{block}
		\begin{block}{Đồng tham chiếu trong khai khoáng ý kiến}
			\begin{itemize}
				\item{Stoyanov và Cardie (2006): Phân giải đồng tham chiếu cho chủ thể ý kiến}
				\item{Ding và Liu (2010): Phân giải đồng tham chiếu đối tượng và thuộc tính \footnotemark}
			\end{itemize}
		\end{block}
		\footnotetext[3]{Xiaowen Ding and Bing Liu. 2010.
			\textit{Resolving Object and Attribute Coreference in Opinion Mining}. 
			In Proceedings of International Conference on Computational Linguistics (COLING-2010). 2010.}		
	\end{frame}

	\section{Phương pháp đề xuất}
	\subsection{Tổng quan quy trình}
	\begin{frame}{Tổng quan quy trình}		
		\begin{figure}[H]
			\LARGE 
			\centering				
			\resizebox{100mm}{!}{% Author: Rasmus Pank Roulund
% \documentclass{minimal}
% \usepackage{tikz}
% \usepackage[utf8]{vietnam}

% \begin{document}
% \usetikzlibrary{arrows,chains,positioning,scopes}

\tikzset{
    block/.style={draw,thick,text width=10em,minimum height=6.5em,minimum width=11em,align=center},
    arrow/.style={->, thick}
}
\begin{tikzpicture}
  {[start chain]
      \node[block,on chain] (N1) {Tập hợp các cụm danh từ};
      \node[block,on chain,join=by {arrow},right=1cm of N1] (N2) {Tiền xử lý};
      \node[block,on chain,join=by {arrow},right=1cm of N2] (N3) {Trích xuất cụm danh từ};
      \node[block,on chain,join=by {arrow},below=1cm of N3] (N4) {Xây dựng dữ liệu học và kiểm tra};
      \node[block,on chain,join=by {arrow},left=1cm of N4] (N5) {Xây dựng bộ phân loại và gom cụm};
      \node[block,on chain,join=by {arrow},left=1cm of N5] (N6) {Các chuỗi đồng tham chiếu};      
    }
      
  \end{tikzpicture}
% \end{document}}
			\caption{Tổng quan quy trình phân giải đồng tham chiếu}	
			\label{fig:generalmodel}						
		\end{figure}
	\end{frame}	

	\subsection{Tiền xử lý}	
	\begin{frame}{Tiền xử lý}		
		\begin{columns}[t]
			\begin{column}{0.4\textwidth}				
			   	\begin{block}{Tiền xử lý văn bản thô}
	   				Sửa lỗi chính tả và một số lỗi nhỏ khác do cách viết không chuẩn mực của người dùng.
				\end{block}
			\end{column}
			\begin{column}{0.6\textwidth}  %%<--- here				
			 	\begin{figure}[H]
					\LARGE 
					\centering				
					\resizebox{65mm}{!}{\input{images/GD_1.pdf_tex}}	
				\end{figure}				
			\end{column}
		\end{columns}
		\begin{columns}[t]
			\begin{column}{\textwidth}				
			   	\begin{block}{Tách câu, tách từ và gán nhãn từ loại}			   		
		   			\color{blue!60!black}\inlineitem{\color{black}Dùng công cụ Stanford \footnotemark}
		   			\color{blue!60!black}\inlineitem{\color{black}Dựa theo Penn Treebank POS Tag \footnotemark}
				\end{block}
			\end{column}			
		\end{columns}
		\footnotetext[4]{http://stanfordnlp.github.io/CoreNLP}
		\footnotetext[5]{http://web.mit.edu/6.863/www/PennTreebankTags.html}
	\end{frame}

	\subsection{Trích xuất cụm danh từ}
	\begin{frame}{Trích xuất cụm danh từ}		
		\begin{columns}[t]
			\begin{column}{0.4\textwidth}
			   	\begin{block}{Tìm các cụm danh từ}
	   				Dùng công cụ CRFChunker \footnotemark.
				\end{block}
				\begin{block}{Lọc lại các cụm danh từ}
			   		Loại một số cụm danh từ vì chúng không thể chỉ về đối tượng hoặc thuộc tính.				
				\end{block}
			\end{column}
			\begin{column}{0.6\textwidth}  %%<--- here
			 	\begin{figure}[H]
					\LARGE 
					\centering				
					\resizebox{65mm}{!}{\input{images/GD_2.pdf_tex}}	
				\end{figure}
			\end{column}
		\end{columns}
		\footnotetext[6]{http://crfchunker.sourceforge.net}
		\begin{columns}[t]
			\begin{column}{\textwidth}
			   	\begin{block}{Gán nhãn cụm danh từ}					
					\textit{Ví dụ:} <0,-1,0 The Note 3> is a lot lighter than <1,-1,0 my HTC EVO>. <2,0,2 It>'s very fast and has <3,0,1 so many features> that <4,-1,0 an IPhone5> can't touch. 
				\end{block}					
			\end{column}			
		\end{columns}
	\end{frame}

	\begin{frame}{Trích xuất cụm danh từ (tt)}
		\begin{figure}[H]
			\centering							
			\includegraphics[scale=0.45]{images/markup_tool}				
			\caption{Công cụ gán nhãn dữ liệu đồng tham chiếu}				
		\end{figure}
	\end{frame}

	\subsection{Xây dựng dữ liệu học và kiểm tra}
	\begin{frame}{Xây dựng dữ liệu học và kiểm tra}				
		\begin{columns}[t]
			\begin{column}{0.4\textwidth}
			   	\begin{block}{Tạo các cặp cụm danh từ}
					Có ít nhất một đối tượng hoặc thuộc tính trong mỗi cặp cụm danh từ được tạo.
				\end{block}
			\end{column}
			\begin{column}{0.6\textwidth}  %%<--- here
			 	\begin{figure}[H]
					\LARGE 
					\centering				
					\resizebox{65mm}{!}{\input{images/GD_3.pdf_tex}}	
				\end{figure}
			\end{column}
		\end{columns}
		\begin{columns}[t]
			\begin{column}{\textwidth}
			   	\begin{block}{Tạo các vectơ đặc trưng}										
					$(<f_{1},f_{2},f_{3},…,f_{n}>, y)$
					\begin{itemize}
						\item{$<f_{1},f_{2},f_{3},…,f_{n}>$: tập đặc trưng}
						\item{$y$: giá trị phân loại}										
					\end{itemize}
				\end{block}					
			\end{column}			
		\end{columns}
	\end{frame}	

	\begin{frame}{Xây dựng dữ liệu học và kiểm tra (tt)}		
		\begin{table}[]		
		\parbox{\textwidth}{
			\centering			
			\fontsize{6pt}{7}\selectfont		
			\begin{tabular}{|l|l|l|}
\hline
Nhóm thuộc tính                                                                               & Thuộc tính                                                                           & Giải thích                                                                                                                                                                                                                                   \\ \hline
\multirow{2}{*}{\begin{tabular}[c]{@{}l@{}}Nhóm liên quan \\ khai khoáng ý kiến\end{tabular}} & Tính nhất quán về ý kiến                                                             & \begin{tabular}[c]{@{}l@{}}Bằng 1 nếu tiền từ và hậu từ có\\ cùng thiên hướng ý kiến. \\ Bằng 0 nếu chúng khác về \\ thiên hướng ý kiến. \\ Nếu không xác định được \\ thiên hướngý kiến cho tiền từ \\ hoặc hậu từ thì bằng 2.\end{tabular} \\ \cline{2-3} 
                                                                                              & \begin{tabular}[c]{@{}l@{}}Sự kết hợp giữa thực thể \\ và từ chỉ ý kiến\end{tabular} & \begin{tabular}[c]{@{}l@{}}Bằng 0,1,2,3,4,10 tùy vào \\ xếp hạng PMI\end{tabular}                                                                                                                                                            \\ \hline
\multirow{6}{*}{\begin{tabular}[c]{@{}l@{}}Nhóm liên quan \\ ngữ pháp\end{tabular}}           & Tiền từ là đại từ                                                                    & \begin{tabular}[c]{@{}l@{}}Bằng 1 nếu tiền từ là đại từ, \\ ngược lại bằng 0\end{tabular}                                                                                                                                                    \\ \cline{2-3} 
                                                                                              & Hậu từ là đại từ                                                                     & \begin{tabular}[c]{@{}l@{}}Bằng 1 nếu hậu từ là đại từ, \\ ngược lại bằng 0\end{tabular}                                                                                                                                                     \\ \cline{2-3} 
                                                                                              & Tính thống nhất về số                                                                & \begin{tabular}[c]{@{}l@{}}Bằng 1 nếu tiền từ và hậu từ \\ thống nhất về số (trong ngữ pháp),\\  ngược lại bằng 0\end{tabular}                                                                                                               \\ \cline{2-3} 
                                                                                              & Từ hạn định                                                                          & \begin{tabular}[c]{@{}l@{}}Bằng 1 nếu hậu từ bắt đầu bằng \\ "the", ngược lại bằng 0\end{tabular}                                                                                                                                            \\ \cline{2-3} 
                                                                                              & Từ chỉ trỏ                                                                           & \begin{tabular}[c]{@{}l@{}}Bằng 1 nếu hậu từ bắt đầu bằng \\ "this", "that", "these", those", \\ ngược lại bằng 0\end{tabular}                                                                                                               \\ \cline{2-3} 
                                                                                              & \begin{tabular}[c]{@{}l@{}}Cả tiền từ và hậu từ \\ đều là danh từ riêng\end{tabular} & \begin{tabular}[c]{@{}l@{}}Bằng 1 nếu tiền từ và hậu từ \\ đều là danh từ riêng, ngược lại \\ bằng 0\end{tabular}                                                                                                                            \\ \hline
\begin{tabular}[c]{@{}l@{}}Nhóm liên quan \\ từ vựng\end{tabular}                             & Tương tự về từ vựng                                                                  & \begin{tabular}[c]{@{}l@{}}Tính tương tự về mặt từ vựng \\ (trùng hoặc gần giống nhau)\end{tabular}                                                                                                                                          \\ \hline
\multirow{3}{*}{Khác}                                                                         & Khoảng cách                                                                          & \begin{tabular}[c]{@{}l@{}}Khoảng cách về câu chứa tiền từ \\ và hậu từ. Nếu chúng nằm cùng \\ một câu thì bằng 0\end{tabular}                                                                                                               \\ \cline{2-3} 
                                                                                              & Từ khóa "is" nằm ở giữa                                                              & \begin{tabular}[c]{@{}l@{}}Bằng 1 nếu có "is" không đi kèm \\ với chỉ định so sánh nào nằm ở \\ giữa tiền từ và hậu từ, ngược lại \\ thì bằng 0\end{tabular}                                                                                 \\ \cline{2-3} 
                                                                                              & Từ khóa "has" nằm ở giữa                                                             & \begin{tabular}[c]{@{}l@{}}Bằng 1 nếu có "has" nằm ở giữa \\ tiền từ và hậu từ, ngược lại thì \\ bằng 0\end{tabular}                                                                                                                         \\ \hline
\end{tabular}	
			\caption{Các đặc trưng được sử dụng trong hệ thống}
		}
		\end{table}
		\hypertarget{features}{}		
	\end{frame}	

	\begin{frame}{Xây dựng dữ liệu học và kiểm tra (tt)}		
		\begin{table}[]		
		\parbox{\textwidth}{
			\centering
			\fontsize{6pt}{7}\selectfont			
			\begin{tabular}{|l|l|l|}
\hline
Nhóm đặc trưng                                                                                & Đặc trưng                                                                            & Giải thích                                                                                                                                                                                                                      \\ \hline
\multirow{3}{*}{\begin{tabular}[c]{@{}l@{}}Nhóm liên quan\\ từ vựng\end{tabular}}             & Giống nhau hoàn toàn                                                                 & \begin{tabular}[c]{@{}l@{}}Bằng 1 nếu NP1 và NP2\\ giống nhau hoàn toàn về mặt\\ từ vựng\end{tabular}                                                                                                                           \\ \cline{2-3} 
                                                                                              & Danh từ chính giống nhau                                                             & \begin{tabular}[c]{@{}l@{}}Bằng true nếu NP1 và NP2 có\\ danh từ chính giống nhau,\\ ngược lại bằng false\end{tabular}                                                                                                                 \\ \cline{2-3} 
                                                                                              & Chuỗi con của nhau                                                                   & \begin{tabular}[c]{@{}l@{}}Bằng true nếu NP1 và NP2 là chuỗi\\ con (cha) của nhau, ngược lại\\ bằng false\end{tabular}                                                                                                                 \\ \hline
\multirow{3}{*}{Khác}                                                                         & Khoảng cách                                                                          & \begin{tabular}[c]{@{}l@{}}Khoảng cách về câu chứa NP1 \\ và NP2. Nếu chúng nằm cùng \\ một câu thì bằng 0\end{tabular}                                                                                                         \\ \cline{2-3} 
                                                                                              & Từ khóa "is" nằm ở giữa                                                              & \begin{tabular}[c]{@{}l@{}}Bằng true nếu có "is" không đi kèm \\ với chỉ định so sánh nào nằm ở \\ giữa NP1 và NP2, ngược lại \\ thì bằng false\end{tabular}                                                                           \\ \cline{2-3} 
                                                                                              & Từ khóa "has" nằm ở giữa                                                             & \begin{tabular}[c]{@{}l@{}}Bằng true nếu có "has" nằm ở giữa \\ NP1 và NP2, ngược lại thì \\ bằng false\end{tabular}                                                                                                                   \\ \hline
\end{tabular}	
			\caption{Các đặc trưng được sử dụng trong hệ thống (tt)}
		}
		\end{table}
		% \hyperlink{results}{\beamerbutton{Kết quả thực nghiệm}}
		\footnotetext[1]{Wee Meng Soon, Hwee Tou Ng, and Daniel Chung Yong Lim. 2001.
	\textit{A machine learning approach to coreference resolution of noun phrases}.
	Computational Linguistics, 27(4):521–544.}
		\footnotetext[2]{Vincent Ng and Claire Cardie. 2002.
	\textit{Improving Machine Learning Approaches to Coreference Resolution}.
	In Proceedings of the 40th Annual Meeting of the Association for Computational Linguistics, pages 104–111.}
	\end{frame}


	\begin{frame}{Xây dựng dữ liệu học và kiểm tra (tt)}
		\begin{block}{Đặc trưng Sự kết hợp giữa thực thể và từ chỉ ý kiến (EOA)}			
			Từ chỉ ý kiến: good, bad, expensive, cheap,... \\ 
			Ví dụ:\\
			\textit{I love the \underline{nokia n95} but not sure how strong \underline{the flash} would be? And also \underline{it} is quite \underline{expensive}, so anyone got any ideas?} \\
			Hiện thực:
				\begin{itemize}
					\item{Xác định từ chỉ ý kiến kèm theo cụm danh từ.}
					\item{Xác định quan hệ từ chỉ ý kiến với các cụm danh từ. 
						\begin{equation*}
						PMI(NP,OW) = log\frac{(P(NP,OW)}{P(NP)\times P(OW)}
						\end{equation*}
				}
					\item{Xếp hạng các mối quan hệ giữa từ chỉ ý kiến với các cụm danh từ.}
				\end{itemize}

		\end{block}	
	\end{frame}

	\begin{frame}{Xây dựng dữ liệu học và kiểm tra (tt)}
		\begin{block}{Đặc trưng Tính nhất quán về ý kiến (SC)}		
			Ví dụ:\\
			\begin{itemize}				
				\item[$\bullet$]{\textit{\underline{The N73} is my favorite. \underline{It} can produce great pictures.}}
				\item[$\bullet$]{\textit{\underline{The K800} is awesome. \underline{That phone} has short battery life.}}
				\item[$\bullet$]{\textit{\underline{The XBR4} is brighter than \underline{the 5080}. Overall, \underline{it} is a great choice.}}
			\end{itemize}
			Hiện thực: \\
				\begin{itemize}
					\item{Xác định thiên hướng ý kiến (tích cực, tiêu cực) của mỗi cụm danh từ.}
					\item{So sánh thiên hướng giữa các cặp cụm danh từ.}
				\end{itemize}
			Giá trị đặc trưng: \\
			\begin{itemize}
				\item{SC = 0: Hai cụm danh từ nằm trong hai câu liên tiếp khác thiên hướng ý kiến}
				\item{SC = 1: Hai cụm danh từ nằm trong hai câu liên tiếp có cùng thiên hướng ý kiến}
				\item{SC = 2: Một trong hai cụm danh từ không xác định được thiên hướng ý kiến hoặc chúng không nằm trong hai câu liên tiếp}
			\end{itemize}									
		\end{block}	
				
	\end{frame}	

	

	\subsection{Tạo bộ phân loại và gom cụm}
	\begin{frame}{Tạo bộ phân loại và gom cụm}		
		\begin{columns}[t]
			\begin{column}{0.4\textwidth}
			   	\begin{block}{Tạo bộ phân loại}
					Giải thuật cây quyết định J48 (trên Weka) được sử dụng để phân loại cho các cặp ứng viên.
				\end{block}				
			\end{column}
			\begin{column}{0.6\textwidth}  %%<--- here
			 	\begin{figure}[H]
					\LARGE 
					\centering				
					\resizebox{65mm}{!}{\input{images/GD_4.pdf_tex}}	
				\end{figure}
			\end{column}
		\end{columns}
		\begin{columns}[t]
			\begin{column}{\textwidth}
			   	\begin{block}{Gom cụm}
			   		(A,B) và (B,C) đồng tham chiếu $\Rightarrow$ Cụm đồng tham chiếu (A,B,C) (nhờ vào tính bắc cầu).
				\end{block}					
			\end{column}			
		\end{columns}
	\end{frame}

	\section{Thực nghiệm và đánh giá}				
		\begin{frame}{Thực nghiệm và đánh giá}			
			\begin{block}{Dữ liệu thực nghiệm}
				Dữ liệu được thu thập từ các bài đánh giá (review) trên \textit{amazon.com} và các bài thảo luận (discussion) từ \textit{http://www.howardforums.com}. Đây là các bài đánh giá và thảo luận về điện thoại. Tập dữ liệu gồm 157 bài, với mỗi bài có trung bình 7-8 câu.
			\end{block}
			\begin{figure}[H]
				\centering				
				\noindent\fbox{
				    \parbox{0.9\textwidth}{
				        The fact that the GS5 is from Samsung makes it the Z2's biggest competitor. The Z2's strongest points are the side-mounted camera button. The Z2 has slightly louder speakers, slightly better battery life than the GS5. It has a fingerprint reader for easier unlocking, a better looking screen and a removable battery. Between the two phones, I'd pick the GS5 for its brighter display and for the ease of use the fingerprint reader brings. That said, watch out because the GS5 doesn't use on-screen menu buttons so handling it can be tricky unless you stick it in a case.
			    	}
				}
				\caption{Ví dụ về một bài đánh giá (review) lấy từ amazon.com}				
			\end{figure}
		\end{frame}		

		\begin{frame}{Thực nghiệm và đánh giá (tt)}			
			\begin{block}{Phương pháp đánh giá}				
				\begin{itemize}
					\item{Kiểm chứng chéo (k-fold cross validation) với k=5}
					\item{Độ đo: Precision, Recall, F-measure}
					\item{Hệ đo: 
						\begin{itemize}
							\item[$\bullet$]{MUC}
							\item[$\bullet$]{B-CUBED}
							\item[$\bullet$]{CEAF-$\Phi_4$}
						\end{itemize}
					}					
				\end{itemize}		
			\end{block}					
		\end{frame}		

		\begin{frame}{Thực nghiệm và đánh giá (tt)}			
			\begin{block}{Các hệ thống được đánh giá}
				\footnotesize						
				\begin{itemize}
					\item{Hệ thống cơ sở: Nhóm liên quan ngữ pháp + Nhóm liên quan từ vựng + Nhóm thuộc tính khác.}
					\item{Hệ thống SC: Hệ thống cơ sở + \textit{Tính nhất quán về ý kiến (SC)}.}
					\item{Hệ thống EOA: Hệ thống cơ sở + \textit{Sự kết hợp giữa thực thể và từ chỉ ý kiến (EOA)}.}
					\item{Hệ thống đầy đủ: Tất cả đặc trưng.}
				\end{itemize}
				% \hyperlink{features}{\beamerbutton{Xem Bảng các đặc trưng}}
				% \hypertarget{results}{}
			\end{block}	
			\begin{table}[]
				\LARGE
				\centering
				\resizebox{\textwidth}{!}{								
				\begin{tabular}{|l|cx{1cm}c|cx{1cm}c|cx{1cm}c|c|c|c|c|c|c|}
				\hline
				                & \multicolumn{3}{c|}{Hệ đo MUC} & \multicolumn{3}{c|}{Hệ đo B3} & \multicolumn{3}{c|}{Hệ đo CEAF-$\Phi_4$} \\ \hline
				                & P        & R        & F        & P        & R        & F       & P         & R         & F         \\  \hline
				Hệ thống cơ sở  &  0.757        &  0.534       &  0.621       &   0.794       &   0.524       & 0.626        & 0.621          & 0.557          &   0.586  \\ \hline
				Hệ thống SC     &   0.742       &  0.630        &  0.680        & 0.735         &  0.608        &  0.661       &  0.666         & 0.593          & 0.627          \\ \hline
				Hệ thống EOA 	&   0.735       &  0.558        &  0.632        & 0.766         &  0.542        &  0.632       &  0.616         & 0.568          & 0.591          \\ \hline
				Hệ thống đầy đủ &  0.730        &   0.632       &  0.676        & 0.724         &  0.610        &   0.658      &  0.661         &  0.594         &  0.626         \\ \hline
				\end{tabular}
				}
				\caption{Kết quả thực nghiệm}
			\end{table}		
		\end{frame}

		\begin{frame}[t]{Thực nghiệm và đánh giá (tt)}								
			\begin{columns}[t]
				\begin{column}{\textwidth}
					\begin{figure}[H] 			
						\centering					
						\includegraphics[scale=0.38]{charts/chart_comparison.pdf}									
					\end{figure} 				
				\end{column}
			\end{columns}
			\begin{columns}[t]
				\begin{column}{\textwidth}
					\begin{block}{Nhận xét}
						\footnotesize		
						\begin{itemize}
							\item{Đặc trưng liên quan đến Khai khoáng ý kiến đã ảnh hưởng tích cực đến kết quả}
							\item{Đặc trưng \textit{Tính nhất quán về ý kiến (SC)} đã có ảnh hưởng tương đối lớn đến hệ thống}
							\item{Đặc trưng \textit{Sự kết hợp giữa thực thể và từ chỉ ý kiến (EOA)} ít có ảnh hưởng tích cực đến hệ thống}
							\item{Việc kết hợp hai đặc trưng mới \textit{Tính nhất quán về ý kiến (SC)} và \textit{Sự kết hợp giữa thực thể và từ chỉ ý kiến (EOA)} vẫn còn gặp vấn đề}
						\end{itemize}			
					\end{block}
				\end{column}				
			\end{columns}						
		\end{frame}	

		\begin{frame}{Thực nghiệm và đánh giá (tt)}
			\begin{block}{Phân tích}
				\begin{itemize}
					\item{Số lượng từ chỉ ý kiến trong tập dữ liệu vẫn chưa nhiều (chỉ có 442 trong số 2634 cụm từ chỉ đối tượng, thuộc tính có từ chỉ ý kiến đi kèm) $\Rightarrow$ Đặc trưng EOA vẫn chưa ảnh hưởng nhiều đến hệ thống.}
					\item{Các từ chỉ ý kiến trong tập dữ liệu vẫn chưa có tính đặc trưng cao, ví dụ như tính từ \textit{good}, \textit{nice}, \textit{better}, \textit{great}.}
					% \item{Theo giả thuyết cặp nào có giá trị đặc trưng EOA càng nhỏ thì càng có khả năng đồng tham chiếu. Tuy nhiên trên thực tế có những trường hợp ngược lại.}					
					\item{Giải thuật tìm thiên hướng ý kiến gắn với đối tượng/thuộc tính được nêu ra trong câu vẫn còn đơn giản và chỉ cho kết quả đúng đối với các câu tương đối đơn giản.}
				\end{itemize}
			\end{block}
		\end{frame}

		\begin{frame}{Thực nghiệm và đánh giá (tt)}			
			\begin{block}{Phân tích (tt)}
				\begin{itemize}
					\item{Mỗi bài đánh giá xuất hiện nhiều đối tượng và thuộc tính $\Rightarrow$ Khó xác định đồng tham chiếu nếu không dựa vào hai đặc trưng liên quan đến khai khoáng ý kiến.}
					\item{Một số trường hợp một đối tượng được diễn tả bởi nhiều tên gọi khác nhau mà không đặc trưng nào trong hệ thống phát hiện ra được. Ví dụ: \textit{the GS3}, \textit{the Samsung S3}, \textit{a new S3 SGH-I747 (US model) ATT phone}.}
					\item{Đặc trưng \textit{Danh từ chính giống nhau} ảnh hưởng nhiều đến độ đúng đắn của hệ thống nhưng có một số trường hợp bắt sai. Ví dụ: \textit{the camera quality} và \textit{the image quality}.}
				\end{itemize}
			\end{block}
		\end{frame}

	\section{Tổng kết}
		\begin{frame}{Tổng kết}			
			\begin{block}{Kết quả đạt được}
				\begin{itemize}
					\item Từ những kiến thức về Phân giải đồng tham chiếu và Khai khoáng ý kiến, đưa ra được phương pháp giải quyết bài toán.
					\item Hiện thực hệ thống dựa trên phương pháp đề xuất, cho kết quả đầu ra khả quan.				
				\end{itemize}
			\end{block}
			\begin{block}{Khó khăn, hạn chế}
				\begin{itemize}
					\item Dữ liệu tự thu thập và tự gán nhãn, công đoạn gán nhãn có thể gặp một số sai sót. 
					\item Do giới hạn về thời gian nên chưa kịp thử nghiệm cho tiếng Việt.				
				\end{itemize}
			\end{block}
		\end{frame}
	
		\begin{frame}{Tổng kết (tt)}			
			\begin{block}{Hướng phát triển}
				\begin{itemize}
					\item Tìm thêm các đặc trưng mới liên quan đến Khai khoáng ý kiến để tăng hiệu suất hệ thống.
					\item Cải thiện đặc trưng \textit{Sự kết hợp giữa thực thể và từ chỉ ý kiến}.
					\item Thử nghiệm cho tiếng Việt.
				\end{itemize}
			\end{block}
		\end{frame}

		\begin{frame}{Hết phần trình bày}
			\Huge
			\centering
			\fontsize{35pt}{35}\selectfont
			\textit{Cảm ơn hội đồng đã lắng nghe!}			
		\end{frame}

	\section{Phụ lục}
		\begin{frame}{Phụ lục: Các hệ đo}
			\begin{block}{Hệ đo MUC}
				\begin{center}
					\inlineitem{\scalebox{1.5}{$P = \frac{\sum \left(|R_i| - |p \left(R_i\right)|\right)}{\sum_{|R_i| - 1}}$}}
					\inlineitem{\scalebox{1.5}{$R = \frac{\sum \left(|S_i| - |p \left(S_i\right)|\right)}{\sum_{|S_i| - 1}}$}}
				\end{center}
			\end{block}
			\begin{block}{Hệ đo CEAF-$\Phi_4$}
				\begin{center}						
					\inlineitem{\scalebox{1.5}{$P = \frac{\Phi \left(g*\right)}{\sum_{S_i \in S*}\Phi_4 \left(S_i, S_i\right)}$}}
					\inlineitem{\scalebox{1.5}{$R = \frac{\Phi \left(g*\right)}{\sum_{R_i \in R*}\Phi_4 \left(R_i, S_j\right)}$}}
				\end{center}
			\end{block}
		\end{frame}

		\begin{frame}{Phụ lục: Các hệ đo (tt)}
			\begin{block}{Hệ đo B-CUBED}
				\begin{center}
					\begin{itemize} 
					\item{\scalebox{1.5}{$P = \frac{1}{n} \sum_{i=1}^{n} \frac{\sum_{j=1}^{|p (S_i)|} |P_{ij}|*\left(|S_i| - |P_{ij}|\right)} {|S_{i}|^2}$}}
					\item{\scalebox{1.5}{$R = \frac{1}{m} \sum_{i=1}^{m} \frac{\sum_{j=1}^{|p (R_i)|} |P'_{ij}|*\left(|S_i| - |P'_{ij}|\right)} {|R_{i}|^2}$}}
					\end{itemize}
				\end{center}
			\end{block}	
			\begin{block}{Công thức tính F}
				\begin{center}									
					\scalebox{1.5}{$F = \frac{2PR}{P+R}$}
				\end{center}
			\end{block}					
		\end{frame}

		\begin{frame}{Phụ lục: Pointwise Mutual Information (PMI)}
			\footnotetext{Fano, R., 1961.
			\textit{Transmission of Information}.
			MIT Press Cambridge, Massachussetts. 
			}
				\begin{center}
						\begin{equation*}
						PMI(NP,OW) = log\frac{(P(NP,OW)}{P(NP)\times P(OW)}
						\end{equation*}
				\end{center}
				\begin{itemize} 
						\item{P(NP): Xác suất cụm danh từ xuất hiện trong tập dữ liệu T.}
						\item{P(OW): Xác suất từ chỉ ý kiến xuất hiện trong tập dữ liệu T.}
						\item{P(NP,OW): Xác suất cụm danh từ và từ chỉ ý kiến cùng xuất hiện trong một câu trong tập dữ liệu T.}
					\end{itemize}
		\end{frame}
\end{document}