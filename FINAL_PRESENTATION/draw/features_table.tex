% \resizebox{0.55\textwidth}{0.4\textheight}{
\begin{tabular}{|l|l|l|}
\hline
Nhóm đặc trưng                                                                                & Đặc trưng                                                                            & Giải thích                                                                                                                                                                                                                      \\ \hline
% \rowcolor[gray]{0.8}
\multirow{2}{*}{\begin{tabular}[c]{@{}l@{}}\textcolor{red}{Nhóm liên quan} \\ \textcolor{red}{khai khoáng ý kiến}\end{tabular}} & \textcolor{red}{Tính nhất quán về ý kiến}                                                             & \begin{tabular}[c]{@{}l@{}}Bằng 1 nếu NP1 và NP2 có\\ cùng thiên hướng ý kiến. \\ Bằng 0 nếu chúng khác về \\ thiên hướng ý kiến. \\ Nếu không xác định được \\ thiên hướng ý kiến cho NP1 \\ hoặc NP2 thì bằng 2.\end{tabular} \\ \cline{2-3} 
                                                                                              & \begin{tabular}[c]{@{}l@{}}\textcolor{red}{Sự kết hợp giữa thực thể} \\ \textcolor{red}{và từ chỉ ý kiến}\end{tabular} & \begin{tabular}[c]{@{}l@{}}Bằng 0,1,2,3,4,10 tùy vào \\ xếp hạng PMI\end{tabular}                                                                                                                                               \\ \hline                                                                                             
\multirow{7}{*}{\begin{tabular}[c]{@{}l@{}}Nhóm liên quan \\ ngữ pháp\end{tabular}}           & NP1 là đại từ                                                                        & \begin{tabular}[c]{@{}l@{}}Bằng true nếu NP1 là đại từ, \\ ngược lại bằng false\end{tabular}                                                                                                                                           \\ \cline{2-3} 
                                                                                              & NP2 là đại từ quan hệ                                                                & \begin{tabular}[c]{@{}l@{}}Bằng 1 nếu NP2 là đại từ quan hệ,\\ ngược lại bằng false\end{tabular}                                                                                                                                    \\ \cline{2-3} 
                                                                                              & NP2 là đại từ                                                                        & \begin{tabular}[c]{@{}l@{}}Bằng true nếu NP2 là đại từ, \\ ngược lại bằng false\end{tabular}                                                                                                                                           \\ \cline{2-3} 
                                                                                              & Tính thống nhất về số                                                                & \begin{tabular}[c]{@{}l@{}}Bằng true nếu NP1 và NP2 \\ thống nhất về số (trong ngữ pháp),\\  ngược lại bằng false\end{tabular}                                                                                                         \\ \cline{2-3} 
                                                                                              & Từ hạn định                                                                          & \begin{tabular}[c]{@{}l@{}}Bằng true nếu NP2 bắt đầu bằng \\ "the", ngược lại bằng false\end{tabular}                                                                                                                                  \\ \cline{2-3} 
                                                                                              & Từ chỉ trỏ                                                                           & \begin{tabular}[c]{@{}l@{}}Bằng true nếu NP2 bắt đầu bằng \\ "this", "that", "these", those", \\ ngược lại bằng false\end{tabular}                                                                                                     \\ \cline{2-3} 
                                                                                              & \begin{tabular}[c]{@{}l@{}}Cả NP1 và NP2 \\ đều là tên riêng\end{tabular}            & \begin{tabular}[c]{@{}l@{}}Bằng true nếu NP1 và NP2 \\ đều là tên riêng, ngược lại \\ bằng false\end{tabular}                                                                                                                          \\ \hline
\end{tabular}