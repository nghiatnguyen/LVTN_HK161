\begin{tabular}{|l|l|l|}
\hline
Nhóm thuộc tính                                                                               & Thuộc tính                                                                           & Giải thích                                                                                                                                                                                                                                   \\ \hline
\multirow{2}{*}{\begin{tabular}[c]{@{}l@{}}Nhóm liên quan \\ khai khoáng ý kiến\end{tabular}} & Tính nhất quán về ý kiến                                                             & \begin{tabular}[c]{@{}l@{}}Bằng 1 nếu tiền từ và hậu từ có\\ cùng thiên hướng ý kiến. \\ Bằng 0 nếu chúng khác về \\ thiên hướng ý kiến. \\ Nếu không xác định được \\ thiên hướngý kiến cho tiền từ \\ hoặc hậu từ thì bằng 2.\end{tabular} \\ \cline{2-3} 
                                                                                              & \begin{tabular}[c]{@{}l@{}}Sự kết hợp giữa thực thể \\ và từ chỉ ý kiến\end{tabular} & \begin{tabular}[c]{@{}l@{}}Bằng 0,1,2,3,4,10 tùy vào \\ xếp hạng PMI\end{tabular}                                                                                                                                                            \\ \hline
\multirow{6}{*}{\begin{tabular}[c]{@{}l@{}}Nhóm liên quan \\ ngữ pháp\end{tabular}}           & Tiền từ là đại từ                                                                    & \begin{tabular}[c]{@{}l@{}}Bằng 1 nếu tiền từ là đại từ, \\ ngược lại bằng 0\end{tabular}                                                                                                                                                    \\ \cline{2-3} 
                                                                                              & Hậu từ là đại từ                                                                     & \begin{tabular}[c]{@{}l@{}}Bằng 1 nếu hậu từ là đại từ, \\ ngược lại bằng 0\end{tabular}                                                                                                                                                     \\ \cline{2-3} 
                                                                                              & Tính thống nhất về số                                                                & \begin{tabular}[c]{@{}l@{}}Bằng 1 nếu tiền từ và hậu từ \\ thống nhất về số (trong ngữ pháp),\\  ngược lại bằng 0\end{tabular}                                                                                                               \\ \cline{2-3} 
                                                                                              & Từ hạn định                                                                          & \begin{tabular}[c]{@{}l@{}}Bằng 1 nếu hậu từ bắt đầu bằng \\ "the", ngược lại bằng 0\end{tabular}                                                                                                                                            \\ \cline{2-3} 
                                                                                              & Từ chỉ trỏ                                                                           & \begin{tabular}[c]{@{}l@{}}Bằng 1 nếu hậu từ bắt đầu bằng \\ "this", "that", "these", those", \\ ngược lại bằng 0\end{tabular}                                                                                                               \\ \cline{2-3} 
                                                                                              & \begin{tabular}[c]{@{}l@{}}Cả tiền từ và hậu từ \\ đều là danh từ riêng\end{tabular} & \begin{tabular}[c]{@{}l@{}}Bằng 1 nếu tiền từ và hậu từ \\ đều là danh từ riêng, ngược lại \\ bằng 0\end{tabular}                                                                                                                            \\ \hline
\begin{tabular}[c]{@{}l@{}}Nhóm liên quan \\ từ vựng\end{tabular}                             & Tương tự về từ vựng                                                                  & \begin{tabular}[c]{@{}l@{}}Tính tương tự về mặt từ vựng \\ (trùng hoặc gần giống nhau)\end{tabular}                                                                                                                                          \\ \hline
\multirow{3}{*}{Khác}                                                                         & Khoảng cách                                                                          & \begin{tabular}[c]{@{}l@{}}Khoảng cách về câu chứa tiền từ \\ và hậu từ. Nếu chúng nằm cùng \\ một câu thì bằng 0\end{tabular}                                                                                                               \\ \cline{2-3} 
                                                                                              & Từ khóa "is" nằm ở giữa                                                              & \begin{tabular}[c]{@{}l@{}}Bằng 1 nếu có "is" không đi kèm \\ với chỉ định so sánh nào nằm ở \\ giữa tiền từ và hậu từ, ngược lại \\ thì bằng 0\end{tabular}                                                                                 \\ \cline{2-3} 
                                                                                              & Từ khóa "has" nằm ở giữa                                                             & \begin{tabular}[c]{@{}l@{}}Bằng 1 nếu có "has" nằm ở giữa \\ tiền từ và hậu từ, ngược lại thì \\ bằng 0\end{tabular}                                                                                                                         \\ \hline
\end{tabular}