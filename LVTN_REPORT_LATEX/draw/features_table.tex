\resizebox{0.9\textwidth}{!}{
\begin{tabular}{|l|l|l|}
\hline
Nhóm đặc trưng                                                                                & Đặc trưng                                                                            & Giải thích                                                                                                                                                                                                                      \\ \hline
\multirow{2}{*}{\begin{tabular}[c]{@{}l@{}}Nhóm liên quan \\ khai khoáng ý kiến\end{tabular}} & Tính nhất quán về ý kiến                                                             & \begin{tabular}[c]{@{}l@{}}Bằng 1 nếu NP1 và NP2 có\\ cùng thiên hướng ý kiến. \\ Bằng 0 nếu chúng khác về \\ thiên hướng ý kiến. \\ Nếu không xác định được \\ thiên hướng ý kiến cho NP1 \\ hoặc NP2 thì bằng 2.\end{tabular} \\ \cline{2-3} 
                                                                                              & \begin{tabular}[c]{@{}l@{}}Sự kết hợp giữa thực thể \\ và từ chỉ ý kiến\end{tabular} & \begin{tabular}[c]{@{}l@{}}Bằng 0,1,2,3,4,10 tùy vào \\ xếp hạng PMI\end{tabular}                                                                                                                                               \\ \hline
\multirow{7}{*}{\begin{tabular}[c]{@{}l@{}}Nhóm liên quan \\ ngữ pháp\end{tabular}}           & NP1 là đại từ                                                                        & \begin{tabular}[c]{@{}l@{}}Bằng true nếu NP1 là đại từ, \\ ngược lại bằng false\end{tabular}                                                                                                                                           \\ \cline{2-3} 
                                                                                              & NP2 là đại từ quan hệ                                                                & \begin{tabular}[c]{@{}l@{}}Bằng 1 nếu NP2 là đại từ quan hệ,\\ ngược lại bằng false\end{tabular}                                                                                                                                    \\ \cline{2-3} 
                                                                                              & NP2 là đại từ                                                                        & \begin{tabular}[c]{@{}l@{}}Bằng true nếu NP2 là đại từ, \\ ngược lại bằng false\end{tabular}                                                                                                                                           \\ \cline{2-3} 
                                                                                              & Tính thống nhất về số                                                                & \begin{tabular}[c]{@{}l@{}}Bằng true nếu NP1 và NP2 \\ thống nhất về số (trong ngữ pháp),\\  ngược lại bằng false\end{tabular}                                                                                                         \\ \cline{2-3} 
                                                                                              & Từ hạn định                                                                          & \begin{tabular}[c]{@{}l@{}}Bằng true nếu NP2 bắt đầu bằng \\ "the", ngược lại bằng false\end{tabular}                                                                                                                                  \\ \cline{2-3} 
                                                                                              & Từ chỉ trỏ                                                                           & \begin{tabular}[c]{@{}l@{}}Bằng true nếu NP2 bắt đầu bằng \\ "this", "that", "these", those", \\ ngược lại bằng false\end{tabular}                                                                                                     \\ \cline{2-3} 
                                                                                              & \begin{tabular}[c]{@{}l@{}}Cả NP1 và NP2 \\ đều là tên riêng\end{tabular}            & \begin{tabular}[c]{@{}l@{}}Bằng true nếu NP1 và NP2 \\ đều là tên riêng, ngược lại \\ bằng false\end{tabular}                                                                                                                          \\ \hline
\multirow{3}{*}{\begin{tabular}[c]{@{}l@{}}Nhóm liên quan\\ từ vựng\end{tabular}}             & Giống nhau hoàn toàn                                                                 & \begin{tabular}[c]{@{}l@{}}Bằng 1 nếu NP1 và NP2\\ giống nhau hoàn toàn về mặt\\ từ vựng\end{tabular}                                                                                                                           \\ \cline{2-3} 
                                                                                              & Danh từ chính giống nhau                                                             & \begin{tabular}[c]{@{}l@{}}Bằng true nếu NP1 và NP2 có\\ danh từ chính giống nhau,\\ ngược lại bằng false\end{tabular}                                                                                                                 \\ \cline{2-3} 
                                                                                              & Chuỗi con của nhau                                                                   & \begin{tabular}[c]{@{}l@{}}Bằng true nếu NP1 và NP2 là chuỗi\\ con (cha) của nhau, ngược lại\\ bằng false\end{tabular}                                                                                                                 \\ \hline
\multirow{3}{*}{Khác}                                                                         & Khoảng cách                                                                          & \begin{tabular}[c]{@{}l@{}}Khoảng cách về câu chứa NP1 \\ và NP2. Nếu chúng nằm cùng \\ một câu thì bằng 0\end{tabular}                                                                                                         \\ \cline{2-3} 
                                                                                              & Từ khóa "is" nằm ở giữa                                                              & \begin{tabular}[c]{@{}l@{}}Bằng true nếu có "is" không đi kèm \\ với chỉ định so sánh nào nằm ở \\ giữa NP1 và NP2, ngược lại \\ thì bằng false\end{tabular}                                                                           \\ \cline{2-3} 
                                                                                              & Từ khóa "has" nằm ở giữa                                                             & \begin{tabular}[c]{@{}l@{}}Bằng true nếu có "has" nằm ở giữa \\ NP1 và NP2, ngược lại thì \\ bằng false\end{tabular}                                                                                                                   \\ \hline
\end{tabular}
}